\documentclass[12pt]{article}
\usepackage[a4paper,margin=1cm,footskip=.5cm]{geometry}
\usepackage[]{authblk}
\usepackage{graphicx}
\usepackage{hanging}
\usepackage{indentfirst}
\usepackage{fancyhdr}
\usepackage{setspace}
\usepackage{enumitem}
\usepackage{multicol}
\usepackage{color}
\usepackage{lipsum}
\usepackage[breaklinks]{hyperref}
\usepackage[all]{hypcap}
\definecolor{darkblue}{rgb}{0.0,0.0,0.5}
%\onehalfspace
\setlength{\evensidemargin}{0in}
\setlength{\oddsidemargin}{0in}
\setlength{\paperheight}{11in}
\setlength{\paperwidth}{8.75in}
\setlength{\tabcolsep}{0in}
\setlength{\textheight}{9in} % This changed from 9in
\setlength{\textwidth}{6.5in}
\setlength{\topmargin}{0.15in}
\setlength{\topskip}{0mm}
\setlength{\voffset}{-0.5in}
\setlength{\parindent}{1cm}
%\setlength{\headheight}{99pt}
\setlength{\headheight}{0.5in}
\parskip = 1mm
\pagestyle{plain}

\definecolor{citescol}{RGB}{73,0,165}
\definecolor{urlscol}{RGB}{3,0,255}
\definecolor{linkscol}{RGB}{187,24,0}
\definecolor{isured}{RGB}{204,0,51}
\definecolor{isukhaki}{RGB}{76,69,43}
\definecolor{coolblack}{rgb}{0.0, 0.18, 0.39}
\let\oldtextbf\textbf
\renewcommand{\textbf}[1]{\textcolor{coolblack}{\oldtextbf{#1}}}

\hypersetup{colorlinks=true,linkcolor=linkscol,citecolor=citescol,urlcolor=urlscol}

\begin{document}

\thispagestyle{fancy}
\begin{flushright}
\today
\end{flushright}
\vspace{2mm}
\begin{flushleft}
\textbf{Course Title:} Biological Data Analysis \\
\textbf{Course Number:} GBIO 408/508\\


\textbf{Course Date:} Fall 2024 \\

\textbf{Course Meeting Times:} TuTh 12:30PM - 3:20PM \\
\textbf{Course Meeting Location:} Biology Building 322 \\
\end{flushleft}

\bigskip

\begin{flushleft}
\textbf{Course Faculty:} Dr. April Wright \\
\textbf{Office:} Meade 113B\\
\textbf{Office Phone:} 5556 \\
\textbf{Email:} april.wright@selu.edu   \\
\textbf{Office Hours:} MW 12-4, and by appointment. Virtual meetings can be booked here: https://calendly.com/april-wright/ \\

\end{flushleft}

\bigskip

\begin{flushleft}

\textbf{Course Description}
In this course, we will explore the fundamentals of managing data and performing analyses computationally. This course is intended for biologists who do not have experience with programming or computational sciences.

\end{flushleft}

\bigskip
\begin{flushleft}

\textbf{Course Objectives}

\begin{itemize}

\item Work with data using programming
\item Make appropriate visualizations of data
\item Create computational reports from raw data
\item Use revision management to track changes to data and code
\item Distribute analyses to colleagues
\item Perform basic statistical analyses in R
\item Query data from the internet to answer questions

\end{itemize}
\end{flushleft}

\bigskip

\begin{flushleft}
\textbf{Assessment}
A grade of `C" or better in this course is required to satisfy the curriculum requirements for the College of Science and Technology. There are a total of 700 points in this course. They are distributed as follows:

\begin{itemize}
\item \textbf{Projects:} 100 pts each
\item \textbf{Homeworks:} 100 pts (10 points each)
\item \textbf{Classroom exercises:} 100 pts
\item \textbf{Presentation:} 100 pts
\end{itemize}

\bigskip

\textbf{Grades will be assigned as follows:}

A: 630-700 points, B: 560-629 pts, C: 490-559 pts, D: 420-489, F: Below 419 pts

\end{flushleft}

\bigskip

\begin{flushleft}
\textbf{Attendance and Make-Up}
\end{flushleft}

 Attendance is expected, and completion grade activities will be turned in almost every class period. Homeworks will be posted via the course Moodle. Homework will be due every Friday on non-exam weeks. Because they will be available for the entire week before they are due, \textbf{no make ups} will be available for assignments unless prior approval is granted. \par
 If you are aware in advance of absences, please let me know. The more information we have, the easier it is for me to accommodate you. \par

\begin{flushleft}
\textbf{Important Dates}
\end{flushleft}

\begin{flushleft}
\textbf{Important Dates}
\end{flushleft}

\begin{itemize}
\item Wednesday, September 18: Academic Checkpoint 1
\item Wednesday, October 16: Academic Checkpoint 2
\item Friday, Nov 1: Withdrawal deadline
\item Friday, December 6: Last day of classes

\end{itemize}

\begin{flushleft}
\textbf{Schedule}
\end{flushleft}

Lecture materials will be posted the day before they are given by 5 pm.

\begin{itemize}
\item Week of Aug. 19: Introduction to R and RStudio; Homework One
\item Week of Aug. 26:  Working with Data I; Homework Two
\item Week of Sept. 2: Working with Data II; Homework Three
\item Week of Sept. 9: Visualization, Project 1 due
\item Week of Sept. 16: Project Structuring; Homework Four
\item Week of Sept. 23: Programming I; Homework Five
\item Week of Sept. 30:  Project Structuring; Homework Six
\item Week of Oct. 7: Revision Management \& Fall Break
\item Week of Oct. 14:  Basic Stats in R; Project II due
\item Week of Oct. 21: Genetic Data; Homework Seven
\item Week of Oct. 28:  Maps and Spatial Data; Homework Eight
\item Week of Nov. 5: R Packages and Ecology; Homework 9
\item Week of Nov. 12:  Maps and Location; Homework 10
\item Week of Nov. 19: Simulations, Project Three
\item Week of Nov. 26: Final Project Presentations
\item Week of Dec. 6:
\item Final:
   Monday, December 9, 12:30-2:3-: Worktime, if you'd like to install R on a personal machine for research use.


\end{itemize}
\end{document}
